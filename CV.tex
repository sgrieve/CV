%-----------------------------------------------------
% Stuart WD Grieve CV
%
% Based on Simon M. Mudd's CV: github.com/simon-m-mudd/CV_SMudd/
%
% This CV is loosely based on the template produced by
% Dario Taraborelli
% URL: http://nitens.org/taraborelli/cvtex
%----------------------------------------------------

%!TEX TS-program = xelatex
%!TEX encoding = UTF-8 Unicode

\documentclass[10pt, a4paper]{article}
\usepackage{fontspec}

% DOCUMENT LAYOUT
\usepackage{geometry}
\geometry{a4paper, textwidth=5.75in, textheight=9.5in, marginparsep=7pt, marginparwidth=.6in}
\setlength\parindent{0in}


% ---- CUSTOM COMMANDS
\chardef\&="E050
\newcommand{\html}[1]{\href{#1}{\scriptsize\textsc{[html]}}}
\newcommand{\pdf}[1]{\href{#1}{\scriptsize\textsc{[pdf]}}}
\newcommand{\doi}[1]{\href{#1}{\scriptsize\textsc{[doi]}}}


% ---- MARGIN YEARS
\usepackage{marginnote}
\newcommand{\amper{}}{\chardef\amper="E0BD }
\newcommand{\years}[1]{\marginnote{\scriptsize #1}}
\renewcommand*{\raggedleftmarginnote}{}
\setlength{\marginparsep}{7pt}
\reversemarginpar

% HEADINGS
\usepackage{sectsty}
\usepackage[normalem]{ulem}
\sectionfont{\mdseries\upshape\Large}
\subsectionfont{\mdseries\scshape\normalsize}
\subsubsectionfont{\mdseries\upshape\large}

\newcommand{\MONTH}{%
  \ifcase\the\month
  \or January% 1
  \or February% 2
  \or March% 3
  \or April% 4
  \or May% 5
  \or June% 6
  \or July% 7
  \or August% 8
  \or September% 9
  \or October% 10
  \or November% 11
  \or December% 12
  \fi}

%columns
\usepackage{multicol}
\setlength{\columnsep}{10pt}
%\def\columnseprulecolor{\color{White}}

% PDF SETUP
% ---- FILL IN HERE THE DOC TITLE AND AUTHOR
\usepackage[bookmarks, colorlinks, breaklinks,
% ---- FILL IN HERE THE TITLE AND AUTHOR
	pdftitle={Stuart W D Grieve - cv},
	pdfauthor={Stuart W. D. Grieve},
]{hyperref}
\hypersetup{linkcolor=blue,citecolor=blue,filecolor=black,urlcolor=blue}

%Colours
\usepackage[usenames,dvipsnames]{xcolor}
\definecolor{Gray}{rgb}{0.3,0.3,0.3}
\definecolor{LightGray}{rgb}{0.6,0.6,0.6}

% Header and footer
\usepackage{fancyhdr}
\pagestyle{fancy}
\fancyhf{}
\rhead{\textcolor{LightGray}{SWD Grieve -- \MONTH \ \the\year}}
\renewcommand{\footrulewidth}{1pt}
\rfoot{\textcolor{LightGray}{Page \thepage}}

%urls
\usepackage{url}

% DOCUMENT
\begin{document}

{\LARGE Stuart William David Grieve}\\
\textcolor{Gray}{\large{PhD Student}}\\[0.1cm]

\begin{multicols}{2}
University of Edinburgh\\
School of GeoSciences\\
Drummond Street\\
Edinburgh, EH8 9XP\\
United Kingdom

\columnbreak

Phone: \texttt{+44~(0)131~650~9170}\\
email: \href{mailto:s.grieve@ed.ac.uk}{s.grieve@ed.ac.uk}\\
Web: \href{http://www.geos.ed.ac.uk/homes/s0675405/}{geos.ed.ac.uk/homes/s0675405/}\\
Blog: \href{http://sgrieve.github.io/}{sgrieve.github.io/}\\
Google Scholar: \href{https://scholar.google.co.uk/citations?user=VwQbAzQAAAAJ&hl=en}{Stuart W D Grieve}\\
Github: \href{https://github.com/sgrieve}{sgrieve}

\end{multicols}

\hrule
\section*{Education}
\noindent
\years{2013--}\textbf{Ph.D. in Global Change} University of Edinburgh\\[0.05cm]
\textit{Using high resolution topography and dated sedimentation rates to constrain the rates of sediment transport and landslide frequency.}\\[0.05cm]
Supervisors: Dr Simon M Mudd and Dr Tristram C Hales (Cardiff University)\\

\years{2011--2012}\textbf{M.Sc. in Geographical Information Science}(Distinction) University of Edinburgh\\[0.05cm]
Thesis Title: \textit{An automated analysis of the southern San Andreas Fault to explore topography’s relationship with tectonics.}\\[0.05cm]
Supervisor: Dr Simon M Mudd\\

\years{2007--2011}\textbf{B.Sc. (Hons.) in Geology and Physical Geography}(2:1) University of Edinburgh\\[0.05cm]
Thesis Title: \textit{The Influence of Climate Change on Landslide Sediment Yields in the Northern Lake District.}\\[0.05cm]
Supervisors: Dr Simon M Mudd\\

\hrule
\section*{Employment}
\noindent

\years{2016} \textbf{Research Assistant}, University of Cardiff\\[0.05cm]
\years{2015} \textbf{GIS Consultant and Field Course Leader}, GeoBus, University of St Andrews\\[0.05cm]
\years{2012--2013} \textbf{GIS Trainee} Forth Crossing Bridge Constructors\\[0.05cm]

\hrule
\section*{Technical Skills}
\noindent

Accomplished programmer comfortable with object orientated concepts and a range of languages \textbf{(C++, Python, Java, Visual Basic, Perl)} and the use of version control \textbf{(git, subversion)} to manage large projects. Maintains documentation for research group’s code base using \textbf{Doxygen} and \textbf{Unix} shell scripting. Extensive experience in desktop \textbf{(ArcGIS, FME, Whitebox, QGIS)} and web based \textbf{(MapBox, Mapguide)} GIS to solve complex spatial problems. Managing large spatial and non-spatial datasets using SQL databases \textbf{(Oracle, PostgreSQL, MySQL, SQLite)}. Processing raw LiDAR point clouds to produce bare earth DEMs.\\[0.05cm]

During Ph.D. has been involved in developing new topographic analysis routines within \textbf{LSDTopoTools}, including the development of new data objects to efficiently analyse drainage basin properties, to identify landslide initiation zones and the integration of the ESRI shapefile format within the software package. Additionally has experience supporting users, in the use of LSDTopoTools, through training and the production of chapters of a user guide produced using \textbf{asciidoctor}.  \\[0.05cm]

\hrule
\section*{Research Interests}
\noindent

My research is focused towards understanding how sediment transport processes are reflected in landscape morphology, particularly how sediment is transported from hillslopes into channels. The mechanisms of this transport range from the motion of individual particles through to large scale slope failures and debris flows.\\[0.05cm]

This work is facilitated through efficient analysis of large, detailed LiDAR datasets and the design of novel algorithms to extract topographic metrics from the land surface. Such work allows repeatable experiments to be performed on topography, facilitating the identification of universal patterns of sediment transport.\\[0.05cm]

\hrule
\section*{Publications}
\noindent

Grieve, S.W.D., Mudd, S.M., Hurst, M.D., 2016. How long is a hillslope? Earth Surf. Process. Landforms. doi:10.1002/esp.3884\\[0.05cm]

Grieve, S.W.D., Mudd, S.M., Hurst, M.D., Milodowski, D.T., 2016. A nondimensional framework for exploring the relief structure of landscapes. Earth Surface Dynamics Discussions 1–41. doi:10.5194/esurf-2015-53\\[0.05cm]

Mudd, S.M., Attal, M., Milodowski, D.T., Grieve, S.W.D., Valters, D.A., 2014. A statistical framework to quantify spatial variation in channel gradients using the integral method of channel profile analysis. J. Geophys. Res. Earth Surf. 119, 2013JF002981. doi:10.1002/2013JF002981\\[0.05cm]

\hrule
\section*{Conference Presentations}
\noindent
\textbf{Bold} indicates presenting author

\subsection*{Invited Talk}
\textbf{Grieve, S.W.D.}, 2015. Reproducible geographic analysis: Insights from geomorphology. Presented at GIS Update, Edinburgh.

\subsection*{Oral Presentations}

\textbf{Grieve, S.W.D.}, Mudd, S.M., Hurst, M.D., 2015. Constraining hillslope sediment flux using high resolution topographic data. Presented at the BSG Annual General Meeting, Southampton.\\[0.05cm]

\textbf{Clubb, F.J.}, Mudd, S.M., Attal, M., Milodowski, D.T., Grieve, S.W.D., 2015. The Relationship between Drainage Density, Erosion Rate, and Hilltop Curvature: Implications for Sediment Transport Processes. Presented at the BSG Annual General Meeting, Southampton.

\subsection*{Poster Presentations}

Mudd, S.M., \textbf{Grieve, S.W.D.}, Milodowski, D.T., Hurst, M.D., Clubb, F.J., Valters, D.A., 2015. LSDTopoToolBox: Open source geomorphology. Presented at the BSG Annual General Meeting, Southampton.\\[0.05cm]

\textbf{Clubb, F.J.}, Mudd, S.M., Attal, M., Milodowski, D.T., Grieve, S.W.D., 2015. The Relationship between Drainage Density, Erosion Rate, and Hilltop Curvature: Implications for Sediment Transport Processes. Presented at the AGU Fall Meeting, San Francisco.\\[0.05cm]

\textbf{Parker, R.N.}, Hales, T.C., Mudd, S.M., Grieve, S.W.D., 2015. Precipitation and soil accumulation history modifies future landslide hazard. Presented at the AGU Fall Meeting, San Francisco.\\[0.05cm]

\textbf{Parker, R.N.}, Hales, T.C., Mudd, S.M., Grieve, S.W.D., 2015. Climate change has limited impact on soil-mantled landsliding. Presented at the EGU General Assembly, Vienna.\\[0.05cm]

\textbf{Grieve, S.W.D.}, Mudd, S.M., Hales, T.C., 2014. How long is a hillslope? Presented at the AGU Fall Meeting, San Francisco.\\[0.05cm]

\textbf{Mudd, S.M.}, Attal, M., Milodowski, D.T., Grieve, S.W.D., Valters, D.A., 2014. A statistical technique for identifying channels of different steepness in transient landscapes. Presented at the EGU General Assembly, Vienna.\\[0.05cm]

\hrule
\section*{Service}
\noindent

\years{2015}\textbf{Currency Reviewer}: Reference Module in Earth Systems and Environmental Sciences, Elsevier.\\[0.05cm]
\years{2014--2015}\textbf{Session Chair} M.Sc. GIS postgraduate conference, University of Edinburgh.\\[0.05cm]

\hrule
\section*{Funding Received}
\noindent

\years{2014}British Society for Geomorphology Student Travel Grant\\
Award: \textbf{£750} \\[0.05cm]

\years{2014}NERC Cosmogenic Isotope Analysis Facility: \textit{Hillslope-channel coupling in a steady-state landscape.}\\
P.I.: Tristam Hales (Cardiff University)\\
Co. I. Simon M. Mudd, Robert Parker (Cardiff University) and Stuart W. D. Grieve.\\
Award: \textbf{£19,320} \\[0.05cm]

\years{2013}Safe Software Grant Program\\
Award: \textbf{Software licence for FME Desktop Edition} \\[0.05cm]

\years{2014}SAAS Postgraduate Students’ Allowances Scheme\\
Award: \textbf{£3400} \\[0.05cm]

\years{2014}University of Edinburgh Postgraduate Bursary\\
Award: \textbf{£1300} \\[0.05cm]

\hrule
\section*{Teaching Experience}
\subsection*{Undergraduate Courses (Course Level)}
\years{2015}Geomorphology, Laboratory Demonstrator and Tutor (2nd year)\\[0.05cm]
\years{2014--2015}Cyprus field course (4th year honours)\\[0.05cm]
\years{2014}Earth Surface Systems Course Assistant (1st year) \\[0.05cm]
\years{2014}Fundamental Methods in Geography, Laboratory and Field Demonstrator (2nd year)\\[0.05cm]
\years{2013--2014}Earth Surface Systems, Laboratory Demonstrator and Tutor (1st year)

\subsection*{Postgraduate Courses (Course Level)}
\years{2014--2015}Object Oriented Software Engineering Principles, Laboratory Demonstrator (M.Sc.)\\[0.05cm]
\years{2014--2015}Object Orientated Software Engineering: Spatial Algorithms, Laboratory Demonstrator (M.Sc.)\\[0.05cm]
\years{2014--2015}Principles of Geographical Information Science, Laboratory Demonstrator (M.Sc.)\\[0.05cm]
\years{2014--2015}Introduction To Spatial Analysis, Laboratory Demonstrator (M.Sc.)\\[0.05cm]
\years{2014--2015}Distributed GIS, Laboratory Demonstrator (M.Sc.)\\[0.05cm]
\years{2014--2015}Spatial Modelling, Laboratory Demonstrator (M.Sc.)\\[0.05cm]
\years{2013--2015}Advanced Spatial Database Methods, Laboratory Demonstrator (M.Sc.)\\[0.05cm]
\years{2013--2015}Further Spatial Analysis, Laboratory Demonstrator (M.Sc.)\\[0.05cm]
\years{2013--2016}Geo-Visualisation, Laboratory Demonstrator (M.Sc.)\\[0.05cm]

\hrule
\section*{Professional Memberships}
\noindent

\years{2014--}American Geophysical Union\\[0.05cm]
\years{2014--}British Society for Geomorphology\\[0.05cm]

\end{document}
